% !TeX root=../main.tex
% در این فایل، عنوان پایان‌نامه، مشخصات خود، متن تقدیمی‌، ستایش، سپاس‌گزاری و چکیده پایان‌نامه را به فارسی، وارد کنید.
% توجه داشته باشید که جدول حاوی مشخصات پروژه/پایان‌نامه/رساله و همچنین، مشخصات داخل آن، به طور خودکار، درج می‌شود.
%%%%%%%%%%%%%%%%%%%%%%%%%%%%%%%%%%%%
% دانشگاه خود را وارد کنید
\university{دانشگاه خواجه نصیرالدین طوسی}
% پردیس دانشگاهی خود را اگر نیاز است وارد کنید (مثال: فنی، علوم پایه، علوم انسانی و ...)
\college{...}
% دانشکده، آموزشکده و یا پژوهشکده  خود را وارد کنید
\faculty{دانشکدهٔ ...}
% گروه آموزشی خود را وارد کنید (در صورت نیاز)
\department{گروه ...}
% رشته تحصیلی خود را وارد کنید
\subject{مهندسی ...}
% در صورت داشتن گرایش، خط زیر را از حالت کامت خارج نموده و گرایش خود را وارد کنید
% \field{گرابش}
% عنوان پایان‌نامه را وارد کنید
\title{قالب آماده برای تدوین پروژه، پایان‌نامه و رساله‌های دانشگاه خواجه نصیرالدین طوسی}
% نام استاد(ان) راهنما را وارد کنید
\firstsupervisor{دکتر راهنمای اول}
\firstsupervisorrank{استاد}
\secondsupervisor{دکتر راهنمای دوم}
\secondsupervisorrank{استادیار}
% نام استاد(دان) مشاور را وارد کنید. چنانچه استاد مشاور ندارید، دستورات پایین را غیرفعال کنید.
\firstadvisor{دکتر مشاور اول}
\firstadvisorrank{استادیار}
\secondadvisor{دکتر مشاور دوم}
% نام داوران داخلی و خارجی خود را وارد نمایید.
\internaljudge{دکتر داور داخلی}
\internaljudgerank{دانشیار}
\externaljudge{دکتر داور خارجی}
\externaljudgerank{دانشیار}
\externaljudgeuniversity{دانشگاه داور خارجی}
% نام نماینده کمیته تحصیلات تکمیلی در دانشکده \ گروه
\graduatedeputy{دکتر نماینده}
\graduatedeputyrank{دانشیار}
% نام دانشجو را وارد کنید
\name{محمدسینا}
% نام خانوادگی دانشجو را وارد کنید
\surname{اله‌کرم}
% شماره دانشجویی دانشجو را وارد کنید
\studentID{0123456}
% تاریخ پایان‌نامه را وارد کنید
\thesisdate{زمستان 1401}
% به صورت پیش‌فرض برای پایان‌نامه‌های کارشناسی تا دکترا به ترتیب از عبارات «پروژه»، «پایان‌نامه» و «رساله» استفاده می‌شود؛ اگر  نمی‌پسندید هر عنوانی را که مایلید در دستور زیر قرار داده و آنرا از حالت توضیح خارج کنید.
%\projectLabel{پایان‌نامه}

% به صورت پیش‌فرض برای عناوین مقاطع تحصیلی کارشناسی تا دکترا به ترتیب از عبارت «کارشناسی»، «کارشناسی ارشد» و «دکتری» استفاده می‌شود؛ اگر نمی‌پسندید هر عنوانی را که مایلید در دستور زیر قرار داده و آنرا از حالت توضیح خارج کنید.
%\degree{}
%%%%%%%%%%%%%%%%%%%%%%%%%%%%%%%%%%%%%%%%%%%%%%%%%%%%
%% پایان‌نامه خود را تقدیم کنید! %%
\dedication
{
{\Large تقدیم به:}\\
\begin{flushleft}{
	\huge
به آنان که با علم خود زندگی آزاد می‌سازند\\
	\vspace{7mm}
}
\end{flushleft}
}
%% متن قدردانی %%
%% این متن را به سلیقه‌ی خود تعییر دهید
\acknowledgement{
اکنون که به یاری پروردگار و یاری و راهنمایی اساتید بزرگ موفق به پایان این رساله شده‌ام وظیفه خود دانشته که نهایت سپاسگزاری را از تمامی عزیزانی که در این راه به من کمک کرده‌اند را به عمل آورم:
در آغاز از استاد بزرگ و دانشمند جناب آقای/سرکار خانم …. که راهنمایی این پایانامه را به عهده داشته‌اند کمال تشکر را دارم.
از جناب آقایان/ خانم‌ها …. که اساتید مشاور این پایانامه بوده‌اند نیز قدردانی می‌نمایم.
از داوران گرامی … که زحمت داوری و تصحیح این پایانامه را به عهده داشتند کمال سپاس را دارم.
خالصانه از تمامی اساتید و معلمان و مدرسانی که در مقاطع مختلف تحصیلی به من علم آموخته و مرا از سرچشمه دانایی سیراب کرده‌اند متشکرم.
از کلیه هم دانشگاهیان و همراهان عزیز، دوستان خوبم خانم‌ها و آقایان …. نهایت سپاس را دارم.

و در پایان این پایان‌نامه را تقدیم می‌کنم به …. که با حضورش و همراهی اش همیشه راه را به من نشان داده و مرا در این راه استوار و ثابت قدم نموده است.
}
%%%%%%%%%%%%%%%%%%%%%%%%%%%%%%%%%%%%
%چکیده پایان‌نامه را وارد کنید
\fa-abstract{
تدوین گزارشی مناسب برای ارائه‌ی دستاوردهای هر پروژه‌ و مراحل رسیدن به آن‌ها لازم است. اگر چه باید تمامی کارهای صورت گرفته در پروژه به شکل مناسب در گزارش بیان گردد اما باید به این نکته نیز توجه شود که از بیان مسائل اضافی که ذهن خواننده را از هدف اصلی دور می‌کند اجتناب شود. این راهنما، علاوه بر ارائه‌ی یک قالب نمونه‌ برای تدوین گزارش پروژه، پایان‌نامه و رسالهٔ دکتری که بر اساس دستور العمال دانشگاه خواجه نصیرالدین طوسی ایجاد شده است، راهنمایی‌هایی نیز برای تدوین یک گزارش مناسب ارائه می‌دهد. برای تهیه‌ی این قالب از کلاس 
\lr{kntu-thesis}
و بستهٔ زی‌پرشین استفاده شده است.

چکیده بخش بسیار مهمی از گزارش است که نمایی کلی از آنچه در گزارش بیان خواهد شد را به خواننده نشان می‌دهد. به طور کلی چکیده باید شامل سه بخش شود: اول از همه باید صورت مسئله به اختصار بیان گردد و سپس مشکلات اصلی که در مسیر پروژه وجود داشته است بیان گردد و در نهایت نیز دستاوردهای حاصل شده از پروژه بیان گردد که تمرکز اصلی نیز برروی بخش سوم می‌باشد. توضیحات باید بیانگر نکات اصلی باشند اما اگر در گزارش روش نوینی برای بار اول ارائه گردیده، بهتر است جزئیات بیشتری از آن بیان گردد.چکیده ترجیحاً‌ یک پاراگراف باشد که شامل حدود ۳۰۰ تا ۵۰۰ کلمه می‌شود. متن چکیده باید روان و سلیس باشد و از جملاتی با معنی و روشن استفاده گردد که خواننده را به خواندن ادامه‌ی گزارش ترغیب کند. چکیده متنی جدای از سایر بخش‌ها است و باید به تنهایی گویا و کامل باشد و از ذکر منابع و ارجاع به بخش‌های دیگر گزارش اجتناب شود. همچنین نداشتن غلط املایی و دستور زبانی در چکیده از اهمیت بالاتری نسبت به سایر بخش‌های گزارش برخوردار است.
کلمات کلیدی که در انتهای چکیده فارسی و انگلیسی آورده می‌شود مبنایی برای طبقه‌بندی گزارش در مراکز اطلاعاتی هستند بنابراین باید  کلمه‌ها یا عباراتی برای آن انتخاب شوند که ماهیت، محتوا و گرایش کار را به وضوح نشان دهند.
}
% کلمات کلیدی پایان‌نامه را وارد کنید
\keywords{حداکثر ۵ کلمه یا عبارت، متناسب با عنوان، قالب پایان‌نامه، لاتک}
% انتهای وارد کردن فیلد‌ها
%%%%%%%%%%%%%%%%%%%%%%%%%%%%%%%%%%%%%%%%%%%%%%%%%%%%%%
